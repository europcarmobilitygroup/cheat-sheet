\documentclass{cheatsheet}
\begin{document}
    \info{
        \medskip
            \includesvg[width=5.25cm]{assets/emobg.svg}
            \bigskip
            \github{europcarmobilitygroup}
    }
    \begin{minipage}[t]{.80\linewidth}
        \bigskip%
        \texttt{
            \Huge
                \color{emobg3}Given-When%
                \color{emobg1}(-And)(-But)%
                \color{emobg3}-Then%
                \color{emobg1}(-Finally)%
            \bigskip
            \newline
            \large\color{emobg3!50} Gherkin is a commonly used language for writing behaviour-driven development (BDD) test descriptions in a human-readable format.
            \bigskip
        }
        \par
        \bigskip%
        \sheetTitle{\color{emobg1} \faLink \href{https://cucumber.io/docs/gherkin/reference/}{keywords used to structure and describe the test steps}}
        \bigskip%

        \bigskip%
        \sheetTag{Given}{100} \sheetTagText{sets up the initial context or preconditions for the test and describes the state of the system before any action is taken (starting point of a scenario)}
        \par%
        \bigskip%

        \bigskip%
        \enspace\sheetTag{When}{100} \sheetTagText{describes the action or event that triggers the scenario and represents the specific action one wants to test (interaction with the system)}
        \par%
        \bigskip%

        \bigskip%
        \enspace\quad\sheetTag{And}{60} \sheetTagText{used to continue the description in the same context. It can follow any of the keywords and is used when multiple actions or conditions need to be specified}
        \par%
        \bigskip%

        \bigskip%
        \enspace\qquad\sheetTag{But}{50} \sheetTagText{similar to "And" and used to continue the description when one needs to highlight a contrasting/alternative action (after "And") or expectation (after "Then")}
        \par%
        \bigskip%

        \bigskip%
        \enspace\sheetTag{Then}{100} \sheetTagText{describes the expected outcome of the action specified in the "When" step. Represents the assertion or verification that the system behaves as expected}
        \par%
        \bigskip%

        \bigskip%
        \sheetTag{Finally}{30} \sheetTagText{not as commonly used as the others and describes the cleanup or post-action steps that need to be taken after the main actions and assertions are complete. It's typically used at the end of the scenario (or in the \inlinecode{afterEach()} or equivalent hook)}
        \par%
        \bigskip%

        \bigskip%
        \bigskip%
        \sheetTag{*}{30} \sheetTagText{the least elegant solution and placeholder replacing any keyword used when we have steps that are a list}
        \par%
        \bigskip%     

    \end{minipage}%
    \begin{minipage}[t]{.20\linewidth}
        \sheetHeader
    \end{minipage}%

    \sheetTextbox{24cm}{10cm}{right}{
        \includegraphics[width=4cm]{assets/cucumber.png}
    }
\end{document}
